\usepackage[margin=0.7in]{geometry}
\usepackage{pgfplots}
\pgfplotsset{compat=1.9}
\usepackage{multirow}
\usepackage{graphicx}
\graphicspath{{fig/}}
\DeclareGraphicsExtensions{.jpg, .jpeg, .png}
\usepackage{
    amsmath,
    amsfonts,
    amssymb,
    amsthm,
    mathtools,
    physics,    % он дает \vectorbold{}
    python}
\usepackage[utf8x]{inputenc} % указать кодировку русского текста
\usepackage[russian]{babel} % указать, что язык текста - русский
\usepackage{fancyhdr}
\usepackage{pgfplots}
\pgfplotsset{compat=1.9}
\usepackage{tikz}
\usepackage{float}
\usepackage{flafter}
% \usepackage[table,xcdraw]{xcolor}
\usepackage{wrapfig}
\usepackage{upgreek}
\usepackage{mathtools}
% \documentclass[xcolor=table]{beamer}

\usepackage{indentfirst} %отступ в новом параграфе в первом абзаце будет сделан, если надо, то юзаю \noindent

\usepackage{caption}

\usepackage{color}   %May be necessary if you want to color links
\usepackage{hyperref}
\hypersetup{
    colorlinks=true, %set true if you want colored links
    linktoc=all,     %set to all if you want both sections and subsections linked
    linkcolor=blue,  %choose some color if you want links to stand out
}