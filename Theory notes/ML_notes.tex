\documentclass[11pt]{article}
\usepackage[margin=0.7in]{geometry}
\usepackage{pgfplots}
\pgfplotsset{compat=1.9}
\usepackage{multirow}
\usepackage{graphicx}
\graphicspath{{fig/}}
\DeclareGraphicsExtensions{.jpg, .jpeg, .png}
\usepackage{
    amsmath,
    amsfonts,
    amssymb,
    amsthm,
    mathtools,
    physics,    % он дает \vectorbold{}
    python}
\usepackage[utf8x]{inputenc} % указать кодировку русского текста
\usepackage[russian]{babel} % указать, что язык текста - русский
\usepackage{fancyhdr}
\usepackage{pgfplots}
\pgfplotsset{compat=1.9}
\usepackage{tikz}
\usepackage{float}
\usepackage{flafter}
% \usepackage[table,xcdraw]{xcolor}
\usepackage{wrapfig}
\usepackage{upgreek}
\usepackage{mathtools}
% \documentclass[xcolor=table]{beamer}

\usepackage{indentfirst} %отступ в новом параграфе в первом абзаце будет сделан, если надо, то юзаю \noindent

\usepackage{caption}

\usepackage{color}   %May be necessary if you want to color links
\usepackage{hyperref}
\hypersetup{
    colorlinks=true, %set true if you want colored links
    linktoc=all,     %set to all if you want both sections and subsections linked
    linkcolor=blue,  %choose some color if you want links to stand out
}        % Подключаемые пакеты
% \pagestyle{fancy}	    % Пользовательские стили


\begin{document}

%Титульный лист
\begin{titlepage}
\begin{center}
\hspace{4.0cm}
\vfill

\huge\textbf{
    Здесь разная информация и всякие мысли,\\
    которые у меня накапливаются по ML, математике и 
    иногда по физике.
}

\vfill

\end{center}
\end{titlepage}

\tableofcontents

\newpage

\section{Математика}

\subsection{Статистика и теорвер}

\url{http://npm.mipt.ru/books/lab-intro/Ch3.html}

- про метод макс правдоподобия, хи квадрат и МНК

\url{https://drive.google.com/file/d/17h2YiHMHxdlnVp_x9_QAfeVhssiiASQE/view?usp=sharing}

- statistics for machine learning

\url{https://drive.google.com/file/d/17h2YiHMHxdlnVp_x9_QAfeVhssiiASQE/view?usp=sharing}

- ссылка на мой summary теорвера для МЛ

\subsection*{Мои замечания всякие:}



\subsection{Принципы работы разных моделей}

\subsubsection{НОРМАЛИЗАЦИЯ - for what???}

\subsection{Оценка качества моделей}

\url{https://drive.google.com/file/d/1yIMdmOYVFaqLo6qkMitDXFquPbyo9_sB/view?usp=sharing}

- статья с архива про оценку качества, все очень подробно

\section{Прога}

\subsection{Мысли про примитивные либы(пандас, склерн и тп)}

\subsection{Бустинг в catboost (и не только)}

\url{https://habr.com/ru/company/otus/blog/527554/}

- супер мега статья на хабре про кэтбуст.

\url{https://github.com/Alphonsce/ML_learning/blob/main/house_prices/tree_model.ipynb}

- Также см мой код на гитхабе (здесь топ 10\% в house prices на бустинге) 

\subsection{Заметки о pytorch}

\subsection{Здесь будет все про разные архитектуры сеток}

\subsection{О железе для МЛ}

\subsection{О jit компиляции в питоне и оптимизации вычислений в нем разными пакетами}

Есть Cython - писать в питоне сишный код

Есть Numba - она при помощи менеджера @jit (контекстный менеджер - это по-сути функция принимающая функцию) позволяет некоторые вещи jit компилировать внутри функции.

Лучше использовать @njit, тогда будет использовать только то, что оптимизировано и свалиаться с ошибкой если перешло к обычному питону.

\subsection{* О параллельных вычислениях}

\section{*О базах данных}

Реляционная БД - это набор отношений - то есть табличка по-сути, в ней нельзя сделать вложенность.

Штуки типо Postgresql и пр - это СУБД, у каждой СУБД +- реализован набор базисных операций, которые умеет делать язык запросов SQL

\section{*Заметки по поводу Julia}

\section{**Физика???}

\section{***Не МЛьная математика?????}

\end{document}